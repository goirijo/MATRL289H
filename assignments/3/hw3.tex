\author{John Goiri}
\date{\today}


\documentclass[12pt, fleqn]{article}
\usepackage{amsmath}
\usepackage{graphicx}
\usepackage{sidecap}
\usepackage{multicol}
\usepackage[small]{caption}
\usepackage{amsfonts}
\usepackage{breqn}
\usepackage{amssymb}
\usepackage[export]{adjustbox}
\usepackage{cleveref}

\renewcommand{\thesubsection}{\thesection.\alph{subsection}}
\renewcommand{\thesubsubsection}{\thesubsection.\alph{subsubsection}}
\begin{document}

\section{Dynamical Matrix}
The functions written to generate the tensor basis for homework 2 have been integrated into new routines to calculate the
dynamical matrix for an arbitrary 2 dimensional structure as a function of force constants and k-vectors.
The calculated dynamical matrix is implemented in the following questions to generate the required dispersion curves
and displacement fields.

\section{Reciprocal Lattice}
The reciprocal lattice for the triangular lattice is shown in \cref{fig:reciprocal}. The vectors for the reciprocal lattice
are perpendicular to the realspace vectors and were computed in one step with
\begin{equation}
    \mathbf{L^\star}=2\pi \left( \mathbf{L}^{-1} \right)^{\intercal}
    \label{recip}
\end{equation}

\begin{figure}[h]
    \begin{center}
        \includegraphics[width=\textwidth]{./reciprocal.eps}
    \end{center}
    \caption{Reciprocal lattice of triangular lattice}
    \label{fig:reciprocal}
\end{figure}

\section{Dispersion curves}
The dynamical matrix and corresponding eigenvalues and eigenvectors were calculated numerically.
A high symmetry path in reciprocal space was traced out to get the values of frequencies at various k-points using arbitrarily selected force constants.

\begin{figure}[h]
    \begin{center}
        \includegraphics[width=\textwidth]{./frequencies.eps}
    \end{center}
    \caption{Dispersion curves for triangular lattice following a high symmetry path in reciprocal space}
    \label{fig:dispersion}
\end{figure}

\section{Displacement fields}
Using the eigenvectors at the high symmetry k-point $\mathbf{M}$, two displacement fields can be generated for the triangular lattice.
Each displacement field corresponds to the activation of a single frequency at $\mathbf{M}$.

\begin{figure}[h]
    \begin{center}
        \includegraphics[width=\textwidth]{./displacement0.eps}
    \end{center}
    \caption{First displacement field at $\mathbf{M}$}
    \label{fig:disp0}
\end{figure}

\begin{figure}[h]
    \begin{center}
        \includegraphics[width=\textwidth]{./displacement1.eps}
    \end{center}
    \caption{Second displacement field at $\mathbf{M}$}
    \label{fig:disp1}
\end{figure}

\end{document}
